\documentclass{article}
\usepackage[margin=1.25in]{geometry}
\usepackage[utf8]{inputenc}
\usepackage[document]{ragged2e}
\usepackage{changepage}
\begin{document}
\begin{adjustwidth}{2.5em}{0pt}
\begin{center}
\Large{UG 12 Meeting \#4 Minutes}\\
\end{center}
\end{adjustwidth}


Fourth Meeting (Wednesday, 11-noon, Week 3, 9th August 2017)
\section{Attendance:}
\subsection*{Present}
\begin{itemize}
\item Adam Davidson, Cyrus Villacampa, James Kortman, Jeremy Hughes, Ziang Chen, Ravi Lam
\end{itemize}
\subsection*{Absent}
\begin{itemize}
\item Linlin Song
\end {itemize}

\section{Details:}
\begin{enumerate}
\item Concern about the lateness of receiving robot kit. Psuedo code good fix for that because we can be flexible. But how to communicate between robot and the server without the robot? One suggestion was to simulate information from robot.
\item We agreed to connect robot to computer and the networking to be done later.
\item There is an EV3 simulator program as well.
\item Need to figure out a general architecture before coding. We'll do that on Friday.
\item Localisation - can syncronise with the map by comparing against known things.
\item Programming is difficult without the robot, so we could work on server side first.
\item Most control was agreed to be on the server side, although we'll have to see what functionality the robot will have. Battery/power consumption on robot is an issue if it's doing lots of computations.
\item Simple reflex commands for the robot is a good idea, so it can not run off the table etc
\item Manual vs autonomous is making things a little tricky code design wise
\item Adam to draft UML for robot side
\end{enumerate}

\section{Actions Points:}
\begin{enumerate}
\item Adam to draft UML for robot side
\end{enumerate}

\end{document}